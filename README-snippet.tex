对于软件工程师来说,文档写作实团队沟通和合作的必备技能,高质量的文档是实现有效沟通的简单方式。

本文向读者介绍一种通过Markdown编写技术文档的方法,本文提供如何利用pandoc将Markdown文档编译成各种形式的输出文档,包括:Mediawiki文档、PDF文档、Word文档等。

本文开源地址:
\url{https://github.com/ismdeep/best-practice-in-se-doc-writing}

\section{Markdown}

Markdown 是一种轻量的标记语言,程序员尤为喜爱 Markdown
文档,因其有着简洁、直观和高效等特点。而且很多本地化或者在线的编辑器都支持
Markdown 的语法。本文的目的就是通过 Markdown
语法编辑文档,产出各种交付的文档。

\section{Mediawiki}

通过 Markdown 转 Mediawiki 文档

\begin{verbatim}
$ pandoc -f markdown -t mediawiki hello.md -o hello.wiki
\end{verbatim}

\section{PDF}

首先通过 Markdown 转 Tex 文档

\begin{verbatim}
$ pandoc -r markdown-auto_identifiers -w latex hello.md -o hello-snippet.tex
\end{verbatim}

\section{DOCX}

\begin{verbatim}
$ pandoc -f markdown -t docx README.md -o README.docx
\end{verbatim}

\section{HTML}

\begin{verbatim}
$ pandoc --standalone \
         --embed-resource \
         --metadata title="<Title>" \
         -c github-markdown.css \
         -f gfm \
         -t html \
         -o README.html \
         README.md
\end{verbatim}
